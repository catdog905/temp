\documentclass{article}
\usepackage{cmap}
\usepackage[utf8]{inputenc}
\usepackage[T2A]{fontenc}
\usepackage[russian]{babel}
\usepackage{listings}


\begin{document}
\lstset{language=Python}

Pandas — это библиотека Python, предоставляющая широкие возможности для анализа данных. Поддерживает форматы: .csv, .tsv.xlsx.\\
 \left{
   \begin{lstlisting}
     import pandas as pd
     import numpy as np
   \end{lstlisting}
\right} - подключение библиотек
   \begin{lstlisting}
      df = pd.read\_csv('../../data/telecom\_churn.csv') - загрузка датасета
      df.head() - вывод первых пяти строк датасета
      \left{
        pd.set\_option('display.max\_colums', 100)
        pd.set\_option('display.max\_rows', 100)
      \right} - вывод большего количества информиции
      df.shape() - вывод размера данных
      df.columns() - вывод названия столбцов
      df.info() - вывод общей информации по датафрейму
      df[column].astype(type) - каст типа колонки
      df.describe() - вывод основных статистических характеристик данных
      \begin{itemize}
        \item{число непропущенных данных по каждому числовому признаку}
        \item{среднее отклонение}
        \item{стандартное отклонение}
        \item{диапазон}
        \item{медиана}
        \item{0.25 и 0.75 квартили}
      \end{itemize}
      df.describe(include=['object', 'bool']) - вывод статистики по нечисловым столбцам
      df[column].value\_counts() - вывод распределения данных
      df[column].value\_counts(normalize=True) - вывод относительные частоты  распределения данных 
      df.sort\_values(by='Total day charge', ascending=False) - сортировка по определённому признаку
      df.sort\_values(by=['Chrun', 'Total day charge'], ascending=[True, False]) - сортировка по нескольким признакам
      df['Churn'].mean() - вывод среднего арифмитического числа
      df[df['Churn'] == 1].mean() - среднее число колонок с маской
      df.loc[0:5, 'State':'Area code'] -срез 5 строк
      df.iloc[0:5, 0:3] - срез 4 строк
      df.apply(np.max) - примининие функции np.max к каждому столбцу или кажлой строке(axis=1)
      \left{
      d = {'No' : False, 'Yes' : True}
      df['International plan'] = df['International plan'].map(d)
      df.head()
      \right} - приминение функции к каждой ячейке столбца
      \left{
      d = {'No' : False, 'Yes' : True}
      df = df.replace({'Voice mail plan': d})
      df.head()
      \right} - анологичная операция
      df.groupby(by=grouping\_columns)[columns\_to\_show].function() - группировка данных
      pd.crosstab(df['Churn'], df['International plan'], normalize=True) - Таблица сопряжённости
      df.pibot\_table(['Total day calls', 'Total eve calls', 'Total night calls'], ['Area code'], aggfunc='mean').head(10)
      df.insert(loc=len(df.columns), column='Total calls', value=total\_calls) loc - номер столбца, после которого нужно вставить данный Series, мы указали len(df.columns), чтобы вставить его в самом конце
      или df['Total charge'] = df['Total day charge'] + df['Total eve charge'] + df['Total night charge'] + df['Total intl charge']
      df = df.drop(['Total charge', 'Total calls'], axis=1) - удаление столбцов 
      df.drop([1, 2]).head() - удаление строк
   \end{lstlisting}
\end{document}
